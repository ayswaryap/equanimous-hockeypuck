Wireless communication is gaining more importance due to its quick and easy accessibility and various other advantages. In our thesis we study a Multi-User Multiple Input Multiple Output \ac{MU-MIMO} system, where multiple base stations \ac{BS}s serves multiple users. The data is transmitted over a shared wireless network but since we have multiple \ac{BS}s frequency reuse schemes are introduced to obtain maximum utilization of resources. The available link has certain restrictions and limitations due to interference from the neighboring \ac{BS} like \ac{ICI} due to frequency reuse. In the \ac{DL}, known as the \ac{MIMO} broadcast channel, the \ac{BS} sends different information streams to the users and in the \ac{UL}, the \ac{BS} receives different information from the users. We consider the transmit precoder design in which a vector of information symbols is multiplied with a precoder matrix before the antenna array transmission. \ac{MU-MIMO} in \ac{DL} is interesting because, \ac{MIMO} sum capacity can scale with the minimum of the number of \ac{BS} antennas and the sum of the number of users times the number of antennas per user. This means that \ac{MU-MIMO} can achieve \ac{MIMO} capacity gains with a multiple antenna \ac{BS} and a bunch of single antenna mobile users.

In the existing problem, the \ac{WSRM} with linear transmit precoding for multicell multi-input single-output \ac{MISO} \ac{DL} is considered. \ac{WSRM} scheme is non convex and there exists beamformer designs which are based on achieving the necessary optimal conditions of the \ac{WSRM} scheme. There are previous papers and results on \ac{WSRM} scheme that they work very close to the optimal design using the iterative algorithm considering Karush-Kuhn-Tucker \ac{KKT} equations. 

In this report, we analyze the existing problem of \ac{WSRM} in \ac{MISO} \ac{DL}. The beamformers are based on \ac{SCA} method. In the existing algorithm we see the approximation of \ac{WSRM} with \ac{SOCP} method. This algorithm takes less time for convergence and also obtain optimal beamformers with the objective of maximum sum rate. The limitations of the existing problem of \ac{WSRM} is overcome with different algorithms which follows the same aim as the existing work of approximaing the \ac{WSRM}. The simulation results shows us that the new algorithm used to overcome the limitation performs better in means of convergence rate.