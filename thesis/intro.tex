\selectlanguage{english}
\par
\thispagestyle{empty}

Wireless communication is gaining more importance due to its quick and easy accessibility and various other advantages like improved data rate and range extensions through multiplexing and diversity schemes. Our day to day life requires wireless services and applications with the advent of smart phones, since the demand is more the rate requirement is increasing exponentially. Availability of the spectrum is limited because of greater number of users and their demands, so our spectral usage needs to be improved. \ac{MIMO} communication provides higher system performance compared to the single antenna systems, more number of users can share the available spectrum \cite{alamouti1998simple}.

Let us consider \ac{MIMO}-\ac{IBC} with multiple \ac{BS}s that can transmit signals to the users in its own cell and causes interference to the neighbouring cell users. Some practical examples that can be used to study \ac{MIMO}-\ac{IBC}, Digital -Subscriber Lines (DSL), Cognitive Radio systems, ad-hoc wireless networks, wireless cellular communication.  We design a linear transmit precoders, which is employed at the \ac{BS} to extract the available signal gains from multiple antenna systems. In order to design efficient precoders the knowledge of \ac{CSIT} is essential, for doing this pilots from each user are sent orthogonally to their respective serving \ac{BS}.

In this thesis we study a \ac{MU-MIMO} system, where multiple \ac{BS}s serves users. The data is transmitted over a shared wireless network but since we have multiple \ac{BS}s frequency reuse schemes are introduced to obtain maximum utilization of resources. The available link has certain restrictions and limitations due to interference from the neighboring \ac{BS} like \ac{ICI} due to frequency reuse. In the \ac{DL}, known as the \ac{MIMO} broadcast channel, the \ac{BS} sends different information streams to the users and in the \ac{UL}, the \ac{BS} receives different information from the users. We consider the transmit precoder design in which a vector of information symbols is multiplied with a precoder matrix before the antenna array transmission. \ac{MU-MIMO} in \ac{DL} is interesting because, \ac{MIMO} sum capacity can scale with the minimum of the number of \ac{BS} antennas and the sum of the number of users times the number of antennas per user. This means that \ac{MU-MIMO} can achieve \ac{MIMO} capacity gains with a multiple antenna \ac{BS} and a bunch of single antenna mobile users.

In the existing problem, the \ac{WSRM} with linear transmit precoding for multicell multi-input single-output \ac{MISO} \ac{DL} is considered. \ac{WSRM} scheme is non convex and there exists beamformer designs which are based on achieving the necessary optimal conditions of the \ac{WSRM} scheme. There are previous papers and results on \ac{WSRM} scheme that they work very close to the optimal design using the iterative algorithm considering Karush-Kuhn-Tucker \ac{KKT} equations. 

In this report, we analyze the existing problem of \ac{WSRM} in \ac{MIMO} \ac{DL}. The beamformers are based on \ac{SCA} method. In the existing algorithm we see the approximation of \ac{WSRM} with \ac{SOCP} method. This algorithm takes less time for convergence and also obtain optimal beamformers with the objective of maximum sum rate. The limitations of the existing problem of \ac{WSRM} is overcome with different algorithms which follows the same aim as the existing work of approximaing the \ac{WSRM}. The simulation results shows us that the new algorithm used to overcome the limitation performs better in means of convergence rate.

In this paper we propose a distributed precoder design algorithm for \ac{WSRM} problem,  that is based on \ac{ADMM} approach, \ac{KKT} based approach for \ac{AP-GP} and iterative minimization of weighted  \ac{MSE}. The proposed distributed algorithm requires only the local channel knowledge and converges to a stationary point of the weighted sum-rate maximization problem. The effectiveness of the proposed algorithm is evaluated in the numerical experiments and is discussed in the simulation results section.


