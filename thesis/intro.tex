\selectlanguage{english}
\par
\thispagestyle{empty}
\acresetall
Wireless communication is gaining more importance due to its quick and easy accessibility and various other advantages like improved data rate and range extensions through multiplexing and diversity schemes. Our day to day life requires wireless services and applications with the advent of smart phones, since the demand is more the rate requirement is increasing exponentially. Availability of the spectrum is limited because of greater number of users and their demands, so our spectral usage needs to be improved. \ac{MIMO} communication provides higher system performance compared to the single antenna systems, more number of users can share the available spectrum \cite{alamouti1998simple}.

Let us consider \ac{MIMO}-\ac{IBC} with multiple \ac{BS}s that can transmit signals to the users in its own cell and causes interference to the neighbouring cell users. Some practical examples that can be used to study \ac{MIMO}-\ac{IBC}, Cognitive Radio systems, ad-hoc wireless networks, wireless cellular communication. For \ac{MIMO} broadcast channels, we know that the \ac{DPC} is the  capacity-achieving scheme \cite{weingarten2004capacity}, however, \ac{DPC} requires high complexity since it uses nonlinear interference cancellation technique, and therefore we go for linear precoding techniques. Let us, consider  the  problem  of  \ac{WSRM}  with  linear  transmit  precoding for \ac{MU}-\ac{MIMO} \ac{DL}. But,  the  \ac{WSRM}  problem considered in this thesis work has shown to be NP-hard \cite{luo2008dynamic},  even for single-antenna receivers. Even though there exists several methods to obtain optimal  beamformers \cite{joshi2012weighted,bjornson2012robust,liu2012achieving},  but they  may not  be  practically  possible to use  since  the  complexity  of  finding  optimal  designs  grows  exponentially with  the problem size. Hence, the need of computationally conducive suboptimal solutions to the WSRM problem still remains, as said in \cite{tran2012fast}.

Since we know that the \ac{WSRM} problem is nonconvex and NP-hard, there exists a class of beamformer designs which are based on achieving the necessary optimal conditions of the \ac{WSRM} problem, as can be seen in, \cite{venturino2010coordinated,ng2010linear,christensen2008weighted,shi2011iterativelyshi2011iteratively}. In  \cite{joshi2012weighted} paper, the authors has numerically shown that the suboptimal designs that achieve the necessary optimal conditions of the \ac{WSRM} problem perform very close to the optimal  design. In  \cite{venturino2010coordinated},  the  iterative  coordinated beamforming  algorithm  was  proposed by  manipulating the \ac{KKT} equations. However, this algorithm is not provably convergent. In \cite{christensen2008weighted,shi2011iterativelyshi2011iteratively},  the  \ac{WSRM}  problem with  joint  transceiver design is  solved using  alternating  optimization  between transmit  and  receive  beamforming.  As  we  show  by  numerical  results,  these  methods  have  a  slower convergence rate compared to our proposed design.

In this thesis we study a \ac{MU-MIMO} system, where multiple \ac{BS}s serves users. The data is transmitted over a shared wireless network but since we have multiple \ac{BS}s frequency reuse schemes are introduced to obtain maximum utilization of resources. The available link has certain restrictions and limitations due to interference from the neighboring \ac{BS} like \ac{ICI} due to frequency reuse. In the \ac{DL}, known as the \ac{MIMO} broadcast channel, the \ac{BS} sends different information streams to the users and in the \ac{UL}, the \ac{BS} receives different information from the users. We consider the transmit precoder design in which a vector of information symbols is multiplied with a precoder matrix before the antenna array transmission. \ac{MU-MIMO} in \ac{DL} is interesting because, \ac{MIMO} sum capacity can scale with the minimum of the number of \ac{BS} antennas and the sum of the number of users times the number of antennas per user. This means that \ac{MU-MIMO} can achieve \ac{MIMO} capacity gains with a multiple antenna \ac{BS} and a bunch of single antenna mobile users.

In this report, we analyze the existing problem of \ac{WSRM} in \ac{MIMO} \ac{DL} framework. The beamformers are designed to maximise the sum rate of all users with minimal or no interference among multiple users. In the existing \ac{WSRM} algorithm in \cite{tran2012fast} we see the approximation of \ac{WSRM} with \ac{SOCP} method. The algorithm proposed in our thesis report takes less time for convergence and also obtain optimal beamformers with the objective of maximum sum rate. The limitations of the existing problem of \ac{WSRM} is overcome with different algorithms which follows the same aim as the existing work of approximating the \ac{WSRM}. The simulation results shows us that the new algorithm is used to overcome the limitation and performs better in means of convergence rate.

The main study and contributions in this thesis is on designing precoders where we propose a centralized precoder for the \ac{WSRM} problem, which is non convex in nature that is solved by using the \ac{SCA} and \ac{AO}. The algorithms that are proposed here is solved by a sequence of convex problem that are obtained by approximating the non convex constraints by the convex ones or obtained by fixing a subset of optimization variable. First approach is obtained by the \ac{AP-GP} approach, while in the second method, \ac{MSE} method is exploited. In the next part we have discussed distributed pecoder design using \ac{ADMM} method. We have also proposed an iterative precoder design by solving the \ac{KKT} expressions for the \ac{AP-GP} and \ac{MSE} reformulation, which needs lesser number of information exchange. The proposed distributed algorithm requires only the local channel knowledge and converges to a stationary point of the weighted sum-rate maximization problem. The effectiveness of the proposed algorithm is evaluated in the numerical experiments and is discussed in the simulation results section.

The rest of the chapters in this thesis are presented as follows,. Background review in chapter 2 covers introduction on \ac{MIMO} system and its capacity, also convex and non convex optimization problems are discussed briefly. In chapter 3 a general precoder design for \ac{MIMO}-\ac{IBC} system problem formulation of the existing and the centralized approach is discussed for the \ac{SINR} relaxtion and \ac{MSE} reformulation. In chapter 4 distributed solutions are presented for \ac{AP-GP} and \ac{MSE} reformulation with and without rate constraint. In the next chapter 5 we discuss about the numerical results and simulation plots for the proposed algorithms. Finally in chapter 6 conclusion and summary is discussed.


\me{\textit{Notations}}: In this thesis report standard notations have been used. Vectors and matrices are represented by bold and Upper case letters, and a transpose of a matrix is shown as \me{(.)^T}. The absolute of a complex number is shown as, \me{|.|}, where as, \me{\|.\|_2} shows the \me{\textit{l}_2} norm, and finally the complex space matrix is shown as \me{\mathbb{C}^{a x b}} where, a and b are the dimensions of the matrix. 
